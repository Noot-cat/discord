\documentclass[fleqn]{ltjsarticle}
\usepackage{amsmath}
\usepackage{amssymb}
\usepackage{tikz}
\usepackage{nccmath}
\usepackage{esint}
\usepackage{mleftright}
\everymath{\displaystyle}
\mleftright

\begin{document}

\begin{flalign*}
  \int \tan x \,dx \\
  \int_{-\sqrt{3}}^{\sqrt{3}} \sqrt{3-x^2}\,dx \\
  \int \log x \,dx \\
  \int_{\frac{\pi}{6}}^{\frac{\pi}{3}} \frac{\log \left(\sin x \right)}{\tan x} \,dx \\
  \int e^x \cos x \,dx \\
  \int \frac{dx}{\sin^3 x} \\
  \int \tan^4 x \,dx \\
  \int \frac{dx}{\sin^2 x} \\
  \int \sqrt{1-x} \,dx \\
  \int_{0}^{\frac{\pi}{2}} \cos^2 x \,dx \\
  \int x^2 \sin x \,dx \\
  \int_{0}^{2 \pi} \sqrt{1+ \cos x} \,dx \\
  \int_{0}^{\frac{\pi}{2}} \sqrt{\frac{1- \cos x}{1+ \cos x}} \,dx \\
  \int_{0}^{\frac{\pi}{3}} \sqrt{1+ \sin x} \,dx \\
  \int_{0}^{1} \frac{dx}{1+x^2} \\
\end{flalign*}
  
\newpage

\begin{flalign*}
  \int_{0}^{1} \frac{dx}{3+x^2} \\
  \int_{0}^{1} \frac{dx}{x^2+x+1} \\
  \int x^x \left(1+ \log x \right) \,dx \\
  \int \frac{dx}{\sqrt{1+x^2}} \\
  \int_{2}^{3} \frac{dx}{\sqrt{x^2-1}} \\
  \int \sqrt{1-e^{-2x}} \,dx \\
  \int_{0}^{1} \frac{dx}{1+x^3} \\
  \int \frac{dx}{\sin^4 x} \\
  \int \frac{dx}{\sqrt{x}+\sqrt{x+2}} \\
  \int_{0}^{e} \frac{e^x}{e^{e-x}+e^x} \,dx \\
  \int_{-1}^{1} \sqrt{1-x^2} \,dx \\
  \int \sqrt{1-e^{-2x}} \,dx \\
  \int \tan^2 x \,dx \\
  \int \tan^3 x \,dx \\
  \int \tan^5 x \,dx \\
\end{flalign*}

\newpage

\begin{flalign*}
  King Property \\
  \int_{a}^{b} f \left(x \right) \,dx &= \int_{a}^{b} f \left(a+b-x \right) \,dx \\
  t&=a+b-x \: と置換する \\
  \,dx &= - \,dt \\
  \therefore \int_{b}^{a} -f \left(t \right) \,dt &= \int_{a}^{b} f \left(t \right) \,dt \\
  &= \int_{a}^{b} f \left(x \right) \,dx \: \left(\because 定積分では変数を変更してもよい \right) \\
  因って、\int_{a}^{b} f \left(x \right) \,dx &= \int_{a}^{b} f \left(a+b-x \right) \,dx \\
\end{flalign*}

\begin{tabular}{|c|c|} \hline
  $x$ & $a \to b$ \\ \hline
  $t$ & $b \to a$ \\ \hline
\end{tabular}

\newpage

\begin{flalign*}
  \int \tan x \,dx &\  \quad \tan の一次式→ \frac{\sin x}{\cos x}で一次式に \\
  \int \tan x \,dx &= \int \frac{\sin x}{\cos x} \,dx \: \quad 分数関数はまず最初に→分母の微分を考える \\
  &= - \log \left\lvert \cos x \right\rvert + C \: \\
\end{flalign*}
\newpage

\begin{flalign*}
  \int_{- \sqrt{3}}^{\sqrt{3}} \sqrt{3-x^2} \,dx & \quad \sqrt{a^2-x^2}を含む→x=a\sin tとおく \\
  x &= \sqrt{3} \sin t \: とする \\
  dx &= \sqrt{3} \cos t \,dt \\
  \int_{- \sqrt{3}}^{\sqrt{3}} \sqrt{3-x^2} \,dx &= \int_{- \frac{\pi}{2}}^{\frac{\pi}{2}} \sqrt{3- \left(\sqrt{3} \sin t \right)^2} \sqrt{3} \cos t \,dt \\
  &= 3 \int_{- \frac{\pi}{2}}^{\frac{\pi}{2}} \sqrt{1 - \sin^2 t} \cos t \,dt \\
  &= 3 \int_{- \frac{\pi}{2}}^{\frac{\pi}{2}} \left\lvert \cos t \right\rvert \cos t \,dt \\
  &= 3 \int_{- \frac{\pi}{2}}^{\frac{\pi}{2}} \cos^2 t \,dt \:\left(\because - \frac{\pi}{2} \leq t \leq \frac{\pi}{2} \Rightarrow 0 \leq \forall \cos t \right) \\
  &= \frac{3}{2} \int_{- \frac{\pi}{2}}^{\frac{\pi}{2}} 1+\cos 2t \,dt \:\left(\because \cos^2 x = \frac{1 + \cos 2x}{2} \:\:\:\: \, c.f. \sin^2 x = \frac{1 - \cos 2x}{2} \right) \\
  &= \frac{3}{2} \left[t + \frac{1}{2} \sin 2t \right]_{- \frac{\pi}{2}}^{\frac{\pi}{2}} \\
  &= 3 \left[t + \frac{1}{2} \sin 2t \right]_{0}^{\frac{\pi}{2}} \\
  &= \frac{3}{2} \pi \\
\end{flalign*}

\begin{tabular}{|c|c|} \hline
  $x$ & $- \sqrt{3} \to \sqrt{3}$ \\ \hline
  $t$ & $- \frac{\pi}{2} \to \frac{\pi}{2}$ \\ \hline
\end{tabular}

\newpage

\begin{flalign*}
  \int \frac{dx}{\sin^2 x} \,dx & \ \sin が\cos だったら都合がいい→\frac{\pi}{2}で\cos に \\ 
  \int \frac{dx}{\sin^2 x} &= \int \frac{dx}{\cos^2 \left(\frac{\pi}{2} - x \right)} \\
  &= - \tan\left(\frac{\pi}{2} - x \right) + C \\
  &= - \frac{1}{\tan x} + C \ \\
\end{flalign*}

\newpage

\begin{flalign*}
  \int_{-1}^{1} \sqrt{1-x^2} \,dx & \quad \sqrt{(\qquad)^2} = |\qquad| としたい. \\
  ここで,1-\sin^2 t=\cos^2 tである. \\
  また,1-x^2 \ge 0⇔-1 \le x \le 1 であるから, x=\sin t とおける. \\
  x &= \sin t \: とする \\
  dx &= \cos t \,dt \\
  \int_{-1}^{1} \sqrt{1-x^2} \,dx &= \int_{- \frac{\pi}{2}}^{\frac{\pi}{2}} \sqrt{1-\sin^2 t} \cos t \,dt \\
  &= \int_{- \frac{\pi}{2}}^{\frac{\pi}{2}} \left\lvert \cos t \right\rvert \cos t \,dt \\
  &= \int_{- \frac{\pi}{2}}^{\frac{\pi}{2}} \cos^2 t \,dt \\
  &= \int_{- \frac{\pi}{2}}^{\frac{\pi}{2}} \frac{1+\cos 2t}{2} \,dt \\
  &= \int_{- \frac{\pi}{2}}^{\frac{\pi}{2}} \left(\frac{1}{2}+\frac{\cos 2t}{2} \right) \,dt \\
  &= \frac{1}{2}\left[t+\frac{1}{2}\sin 2t \right]_{- \frac{\pi}{2}}^{\frac{\pi}{2}} \\
  &= \frac{\pi}{2}
\end{flalign*}

\begin{tabular}{|c|c|} \hline
  $x$ & $-1 \to 1$ \\ \hline
  $t$ & $-\frac{\pi}{2} \to \frac{\pi}{2}$ \\ \hline
\end{tabular}

\newpage

\begin{flalign*}
  \int_{0}^{1} \frac{\,dx}{1+x^2} \quad 分母を単項式にしたい. \\
  ここで,1+\tan^2 t=\frac{1}{\cos^2 t}である.また,(\tan t)'=\frac{1}{\cos^2 t}であるから,x=\tan tとする\\
  x &= \tan t \: とおく \\
  1+x^2 &= \frac{1}{\cos^2 t} \\
  dx &= \frac{dt}{\cos^2 t} \\
  \int_{0}^{1} \frac{dx}{1+x^2} &= \int_{0}^{\frac{\pi}{4}} \frac{\frac{1}{\cos^2 t}}{\frac{1}{\cos^2 t}} \,dt \\
  &= \int_{0}^{\frac{\pi}{4}} \,dt \\
  &= \frac{\pi}{4} \\
\end{flalign*}

\begin{tabular}{|c|c|} \hline
  $x$ & $0 \to 1$ \\ \hline
  $t$ & $0 \to \frac{\pi}{4}$ \\ \hline
\end{tabular}

\newpage

\begin{flalign*}
  \int \tan^4 x \,dx \quad 三角関数は2次ごとに分割 \\
  &= \int \tan^2 x \tan^2 x \,dx \quad \tan^2 は\cos^2 \\
  &= \int \tan^2 x \left(\frac{1}{\cos^2 x}-1 \right) \,dx \\
  &= \int \frac{\tan^2 x}{\cos^2 x} \,dx - \int \tan^2 x \,dx \\
  &= \frac{1}{3}\tan^3 x -\tan x + x +C \\
\end{flalign*}

\newpage

\begin{flalign*}
  \int \log x \,dx \quad \log を含む→部分積分 \\
  &= \int 1 \cdot \log x \,dx \\
  &= x\log x -\int x \cdot \frac{1}{x} \,dx \\
  &= x\log x -x +C \\
\end{flalign*}

\newpage

\begin{flalign*}
  \int_{0}^{e} \frac{e^x}{e^{e-x}+e^x} \,dx & \quad a+b-xを含む,下端が0,指数関数を含む式→KingProperty \\
  \int_{0}^{e} \frac{e^x}{e^{e-x}+e^x} \,dx
  &= \int_{0}^{e} \frac{e^{e-x}}{e^x+e^{e-x}} \,dx \\
  I &= \int_{0}^{e} \frac{e^x}{e^{e-x}+e^x} \,dx \: とすると \\
  2I &= \int_{0}^{e} \frac{e^x}{e^{e-x}+e^x} \,dx + \int_{0}^{e} \frac{e^{e-x}}{e^x+e^{e-x}} \,dx \\
  &= \int_{0}^{e} \frac{e^x+e^{e-x}}{e^x+e^{e-x}} \,dx \\
  &= \int_{0}^{e} \,dx \\
  &= e \\
  \therefore I &= \frac{e}{2} \\
\end{flalign*}

\begin{flalign*}
\int_{0}^{e} \frac{e^x}{e^{e-x}+e^x} \,dx & \quad 指数関数の定数部分は分離 \\
&=\int_{0}^{e} \frac{e^x}{e^e \cdot e^{-x}+e^x} \,dx \\
&指数の符号は揃える,指数関数の分数関数→分母分子にe^xをかける\\
&指数関数の分数関数→微分接触をつくる \\
&=\frac{1}{2}\int_{0}^{e}\frac{2e^{2x}}{e^e+e^{2x}} \\
&=\frac{1}{2}\left[\log (e^e+e^{2x})\right]_{0}^{e} \quad (\because e^e+e^{2x} \ge 0) \\
&=\frac{1}{2}\log \frac{e^e+e^{2e}}{e^e+1} \\
&=\frac{1}{2}\log \frac{e^e(1+e^{e})}{e^e+1} \\
&=\frac{e}{2}
\end{flalign*}

\newpage

\begin{flalign*}
  \int \frac{dx}{\sin^4 x} \quad f(\sin^2x,\cos^2x,\tan x)→g(\tan x) \cdot \frac{1}{\cos^2x}に \\
  \int \frac{dx}{\sin^4 x} &= \int \frac{dx}{\tan^4 x \cos^4 x} \\
  &= \int \frac{1}{\tan^4 x} \left(1+\tan^2 x \right)\frac{dx}{\cos^2 x} \\
  t &= \tan x \: とする \\
  dt &= \frac{dx}{\cos^2 x} \\
  \int \frac{1}{\tan^4 x} \left(1+\tan^2 x \right)\frac{dx}{\cos^2 x} &= \int \frac{1}{t^4} \left(1+t^2 \right) \,dt \\
  &= -\frac{1}{3t^3}-\frac{1}{t}+C \\
  &= -\frac{1}{3\tan^3 x}-\frac{1}{\tan x}+C \\
\end{flalign*}

\newpage

\begin{flalign*}
  \int \tan^3 x \quad 三角関数は2次ごとに分割 \,dx \\
  \int \tan^3 x \,dx &= \int \tan x \cdot \tan^2 x \,dx\quad\tan^2 xは\cos^2x \\
  &= \int \tan x \left(\frac{1}{\cos^2 x}-1 \right) \,dx \\
  &= \int \frac{\tan x}{\cos^2 x} \,dx - \int \tan x \,dx \\
  &= \frac{1}{2}\tan^2 x + \log \left\lvert \cos x \right\rvert +C \\
\end{flalign*}

\newpage

\begin{flalign*}
  \int \tan^2 x \,dx\quad \tan^2 xは\cos^2x\\
  \int \tan^2 x \,dx &= \int \left(\frac{1}{\cos^2 x}-1 \right) \,dx \\
  &= \tan x -x +C \\
\end{flalign*}

\newpage

\begin{flalign*}
  \int_{0}^{\frac{\pi}{2}} \cos^2 x \,dx \quad 積分区間の下端が0,三角関数の式で積分区間が{\frac{n\pi}{2}}→King\;Property\\
  I &= \int_{0}^{\frac{\pi}{2}} \sin^2 x \,dx \\
  &= \int_{0}^{\frac{\pi}{2}} \cos^2 x \,dx \\
  2I &= \int_{0}^{\frac{\pi}{2}} \,dx \\
  &= \frac{\pi}{2} \\
  \therefore I &= \frac{\pi}{4} \\
\end{flalign*}

\newpage

\begin{flalign*}
  \int \frac{\log \left(\sin x \right)}{\tan x} & \,dx \quad \tan の一次式→\frac{\sin}{\cos},真数,位相が複雑→=tとして微分接触を疑う\\
  \int \frac{\log \left(\sin x \right)}{\tan x} \,dx &= \int \frac{\cos x \log \left(\sin x \right)}{\sin x} \,dx \\
  &= \int \frac{\log t}{t} \,dt \\
  &= \frac{1}{2}\left(\log t \right)^2 +C \\
  &= \frac{1}{2}\left(\log \left(\sin x \right) \right)^2 +C \\
\end{flalign*}

\newpage

\begin{flalign*}
  \int \frac{dx}{\sqrt{1+x^2}} \\
  x &= \sinh t \\
  dx &= \cosh t \,dt \\
  \int \frac{dx}{\sqrt{1+x^2}} &= \int \frac{1}{\sqrt{1+\sinh^2 t}}\cosh t \,dt \\
  &= \int \,dt \\
  &= t+C \\
  &= \log \left(x+\sqrt{1+x^2}\right)+C \\
\end{flalign*}

\newpage

\begin{flalign*}
  \int \tan^5 x \,dx &= \int \tan^3 x \left(\frac{1}{\cos^2 x}-1 \right) \,dx \\
  &= \frac{1}{4}\tan^4 x - \frac{1}{2}\tan^2 x + \log \left\lvert \cos x \right\rvert +C \\
\end{flalign*}

\newpage

\begin{flalign*}
  \int \frac{dx}{\cos^4 x} & \quad 三角関数は2乗ごとに分割 \\
  \int \frac{dx}{\cos^4 x} &= \int \frac{1}{\cos^2 x} \cdot \frac{1}{\cos^2 x} \,dx \quad \cos^2 xは\tan^2 x \\
  &= \int \left(1+\tan^2 x \right) \frac{dx}{\cos^2 x} \\
  &= \frac{1}{3}\tan^3 x + \tan x +C \\
\end{flalign*}

\newpage

\begin{flalign*}
  \int_{0}^{1} \frac{dx}{x^2+3} \quad x^2+a^2を含む→x=a\tan t\\
  x &= \sqrt{3}\tan t \: とおく \\
  dx &= \frac{\sqrt3 dt}{\cos^2 t} \\
  \int_{0}^{1} \frac{dx}{x^2+3} &= \int_{0}^{\frac{\pi}{6}} \frac{\frac{\sqrt{3}}{\cos^2 t}}{3\tan^2 t +3} \,dt \\
  &= \int_{0}^{\frac{\pi}{6}} \frac{\frac{\sqrt{3}}{\cos^2 t}}{3\left(\frac{1}{\cos^2 t}\right)} \,dt \\
  &= \frac{1}{\sqrt{3}} \int_{0}^{\frac{\pi}{6}} \,dt \\
  &= \frac{\pi}{6\sqrt{3}} \\
\end{flalign*}

\begin{tabular}{|c|c|} \hline
  $x$ & $0 \to 1$ \\ \hline
  $t$ & $0 \to \frac{\pi}{6}$ \\ \hline
\end{tabular}

\newpage

\begin{flalign*}
  \int x^2 \sin x \,dx \quad x^nf(x)→(瞬間)部分積分\\
  \int x^2 \sin x \,dx &= x^2\sin x+2x\cos x -2\sin x+C \\
\end{flalign*}

\begin{tabular}{c c c}
  $+$ & $x^2$ & $\sin x$ \\
  $-$ & $2x$  & $-\cos x$ \\
  $+$ & $2$   & $-\sin x$ \\
\end{tabular}

\newpage

\begin{flalign*}
  &→分母分子に同じもの(\cos x)をかける(分母を優先)\\
  \int \frac{dx}{\cos^3 x} 三角関数は2乗に強い,三角関数や指数関数の分数関数,分母を優先 \\
  \int \frac{dx}{\cos^3 x} &= \int \frac{\cos x}{\cos^4 x} \,dx \\
  &= \int \frac{\cos x}{\left(1-\sin^2 x \right)^2} \,dx \\
  \frac{1}{1-t^2} &= \frac{1}{2}\left(\frac{1}{1+t} + \frac{1}{1-t} \right) \\
  \frac{1}{\left(1-t^2 \right)^2} &= \left(\frac{1}{2}\left(\frac{1}{1+t} + \frac{1}{1-t}\right) \right)^2 \\
  &= \frac{1}{4}\left(\frac{1}{\left( 1+t \right)^2} + \frac{1}{\left( 1-t \right)^2} + \frac{2}{1-t^2} \right) \\
  &= \frac{1}{4}\left(\frac{1}{\left(1+t\right)^2} + \frac{1}{\left(1-t\right)^2} + \frac{1}{1+t} + \frac{1}{1-t} \right) \\
  \int \frac{dt}{\left(1-t^2 \right)^2} \\
  &= \frac{1}{4} \int \left(\frac{1}{\left(1+t\right)^2} + \frac{1}{\left(1-t\right)^2} + \frac{1}{1+t} + \frac{1}{1-t} \right) \,dt \\
  &= \frac{1}{4} \left( \frac{1}{1-t} - \frac{1}{1+t} + \log \left\lvert \frac{t+1}{t-1} \right\rvert \right) +C \\
  &= \frac{1}{4} \left( \frac{1}{1-\sin x} - \frac{1}{1+\sin x} + \log \left\lvert \frac{\sin x+1}{\sin x-1} \right\rvert \right) +C \\
  &= \frac{1}{4} \left( \frac{1}{1+t} - \frac{1}{1-t} + \log \left\lvert \frac{1+t}{1-t} \right\rvert \right) +C \\
  &= \frac{1}{4} \left( \frac{1}{1+\sin x} - \frac{1}{1-\sin x} + \log \frac{1+\sin x}{1-\sin x} \right\rvert \right) +C \\
\end{flalign*}

\newpage

\begin{flalign*}
  \int \frac{dx}{\sin^4 x} \quad f(\sin^2x,\cos^2x,\tan x)→t=\tan x\\
  t &= \tan x \\
  \cos^2 x &= \frac{1}{1+\tan^2 x} \\
  &= \frac{1}{1+t^2} \\
  \sin^2 x &= \cos^2 x \tan^2 x \\
  &= \frac{t^2}{1+t^2} \\
  dt &= \frac{dx}{\cos^2 x} \\
  dx &= \frac{dt}{1+t^2} \\
  \int \frac{dx}{\sin^4 x} &= \int \frac{\left( 1+t^2 \right)^2}{t^4} \cdot \frac{dt}{1+t^2} \\
  &= \int \frac{1+t^2}{t^4} \,dt \\
  &= \int \frac{dt}{t^4} + \int \frac{dt}{t^2} \\
  &= -\frac{1}{3t^3} - \frac{1}{t} +C \\
  &= -\frac{1}{3\tan^3 x} - \frac{1}{\tan x} +C
\end{flalign*}

\newpage

\begin{flalign*}
  \int_{0}^{2\pi} \sqrt{1+\cos x} \,dx \quad \sqrt{(\qquad)^2} &=|\qquad| としたい.\\
  ここで,1+\cos x=2\cos^2 \frac{x}{2}である.\\
  \int_{0}^{\frac{\pi}{3}} \sqrt{1+\cos x} \,dx
  &= \int_{0}^{\frac{\pi}{3}} \sqrt{2\cos^2 \frac{x}{2}} \,dx \\
  &= \sqrt{2} \int_{0}^{\frac{\pi}{3}} \left\lvert \cos \frac{x}{2} \right\rvert \,dx \\
  &= 2\sqrt{2} \left[\sin \frac{x}{2} \right]_{0}^{\frac{\pi}{3}} \\
  &= \sqrt2 \\
\end{flalign*}

\newpage

\begin{flalign*}
  \int_{0}^{\frac{\pi}{3}} \sqrt{1+\sin x} \,dx \quad \sin が\cos だと都合がいい→\frac{\pi}{2}で\cos に \\
  \int_{0}^{\frac{\pi}{3}} \sqrt{1+\sin x} \,dx \quad \sin が\cos だと都合がいい→\frac{\pi}{2} \\
  \int_{0}^{\frac{\pi}{3}} \sqrt{1+\sin x} \,dx &= \int_{0}^{\frac{\pi}{3}} \sqrt{1+\cos \left(\frac{\pi}{2}-x\right)} \,dx \\
  &= \sqrt{2} \int_{0}^{\frac{\pi}{3}} \left\lvert \cos \left( \frac{\pi}{4}-\frac{x}{2} \right) \right\rvert \,dx \\
  &= -2\sqrt{2} \left[ \sin \left( \frac{\pi}{4}-\frac{x}{2} \right) \right]_{0}^{\frac{\pi}{3}} \\
  &= -2\sqrt{2} \left( \frac{\sqrt{6}-\sqrt{2}}{4} - \frac{1}{\sqrt{2}} \right) \\
  &= 3-\sqrt{3} \\
\end{flalign*}

\begin{flalign*}
\int\frac{\cos^2x}{\sin x-1}\,dx &=\int\frac{(1-\sin x)(1+\sin x)}{\sin x-1}dx\\
&=-\int(1+\sin x)dx\\
&=\cos x-x+C
\end{flalign*}

\begin{flalign*}
\int \cos x \sin x \cos 2x \,dx&=\frac{1}{2}\int \sin 2x \cos 2x \\
&=\frac{1}{4}\int\sin 4x\,dx\\
&=-\frac{1}{16}\cos 4x+C
\end{flalign*}
\begin{flalign*}
  \int_{0}^{\frac{\pi}{2}} \sin 6x\cos 4x\,dx&=\frac{1}{2}\int_{0}^{\frac{\pi}{2}}(\sin 10x+\sin2x)\,dx\\
  &=\frac{1}{2}\left[-\frac{1}{10}\cos 10x-\frac{1}{2}\cos 2x\right]_{0}^{\frac{\pi}{2}}\\
  &=\frac{3}{5}
\end{flalign*}

\newpage

\begin{flalign*}
  \int \frac{\sin x\cos x}{1+\sin^2 x} \,dx &= \frac{1}{2}\log (1+\sin^2 x)+C \quad (\because 1+\sin^2 x \ge 0)
  \int_{0}^{\frac{\pi}{4}} \frac{\cos x}{\cos x + \sin x} \,dx \\
  分数関数なので,取り敢えずまず分母の微分を考えておく. \\
  (\cos x + \sin x)'=\cos x -\sin x \\
  \quad定積分の値が求めやすい関数になるように勝手に定積分を足す \\
  \int_{0}^{\frac{\pi}{4}} \frac{\cos x}{\cos x + \sin x} \,dx + \int_{0}^{\frac{\pi}{4}} \frac{\sin x}{\cos x + \sin x} \,dx
  =\frac{\pi}{4} \\
  \int_{0}^{\frac{\pi}{4}} \frac{\cos x}{\cos x + \sin x} \,dx =I,\int_{0}^{\frac{\pi}{4}} \frac{\sin x}{\cos x + \sin x} \,dx = J \quad とすると \\
  I+J=\frac{\pi}{4} \\
  IとJを求めるために,IとJを組み合わせて他に積分しやすい形をつくりたい. \\
  ここで,(\cos x + \sin x)'=\cos x -\sin x であるから,I-Jが簡単に求められる. \\
  I-J=\int_{0}^{\frac{\pi}{4}} \frac{\cos x}{\cos x + \sin x} \,dx - \int_{0}^{\frac{\pi}{4}} \frac{\sin x}{\cos x + \sin x} \,dx \\
  &=\left[\log |\cos x + \sin x|\right]_{0}^{\frac{\pi}{4}} \\
  &=\left[\log |\sqrt 2 \sin (x+ \frac{\pi}{4})|\right]_{0}^{\frac{\pi}{4}} \\
  &=\frac{1}{2}\log 2 \\
  \therefore I=\frac{\pi}{8}+\frac{1}{4}\log 2
\end{flalign*}

\begin{flalign*}
  \int \frac{dx}{\sqrt{x}+\sqrt{x+2}} &= \int \frac{\sqrt{x+2}-\sqrt{x}}{2} \,dx \\
  &= \frac{1}{3}\left(x+2\right)^{\frac{3}{2}} -\frac{1}{3}x^{\frac{3}{2}}+C \\
\end{flalign*}

\newpage

\begin{flalign*}
  \int_{0}^{\frac{\pi}{2}} \sqrt{\frac{1-\cos x}{1+\cos x}} \,dx &= \int_{0}^{\frac{\pi}{2}} \sqrt{\tan^2 \frac{x}{2}} \,dx \\
  &= \int_{0}^{\frac{\pi}{2}} \left\lvert \tan \frac{x}{2} \right\rvert \,dx \\
  &= -2 \left[\log \left\lvert \cos \frac{x}{2} \right\rvert \right]_{0}^{\frac{\pi}{2}} \\
  &= -2 \log \frac{1}{\sqrt{2}} \\
  &= \log 2 \\
\end{flalign*}

\newpage

\begin{flalign*}
  \int_{0}^{1} \frac{dx}{1+x^3} & \quad 分母が因数分解できる→BBB\\
  \int_{0}^{1} \frac{dx}{1+x^3} &= \frac{1}{3} \int_{0}^{1} \left(\frac{1}{1+x}-\frac{x-2}{x^2-x+1}\right) \,dx \\
  \frac{1}{3} \int_{0}^{1} \frac{dx}{1+x} 
  &= \frac{1}{3}\left[\log |1+x|\right]_{0}^{1} \\
  &= \frac{1}{3} \log 2 \\
  \frac{1}{3} \int_{0}^{1} \frac{x-2}{x^2-x+1} \,dx & \quad 分母が2次式で分子が1次式→微分接触をつくる \\
  \frac{x-2}{x^2-x+1} 
  &= \frac{1}{2}\left(\frac{2x-4}{x^2-x+1}\right) \\
  &= \frac{1}{2}\left(\frac{2x-1}{x^2-x+1}-\frac{3}{x^2-x+1}\right) \\
  &※xの係数を定数倍で合わせないと修正項にxが残ってしまう \\
  \frac{1}{3} \int_{0}^{1} \frac{x-2}{x^2-x+1} \,dx 
  &= \frac{1}{6} \int_{0}^{1} \frac{2x-1}{x^2-x+1} - \frac{1}{6}\int_{0}^{1}\frac{3}{x^2-x+1} \,dx \\
  \frac{1}{6} \int_{0}^{1} \frac{2x-1}{x^2-x+1} \,dx &= \frac{1}{6}\left[\log \left\lvert x^2-x+1 \right\rvert \right]_{0}^{1} \\
  &= 0 \\
  \frac{1}{2} \int_{0}^{1} \frac{dx}{\left(x-\frac{1}{2}\right)^2+\frac{3}{4}} \\
  x-\frac{1}{2} &= \frac{\sqrt{3}}{2}\tan t \\
  dx &= \frac{\sqrt{3}}{2\cos^2 t} \,dt \\
  \frac{1}{2} \int_{0}^{1} \frac{dx}{\left(x-\frac{1}{2}\right)^2+\frac{3}{4}} &= \frac{1}{2} \cdot \frac{4}{3} \cdot \frac{\sqrt{3}}{2} \int_{- \frac{\pi}{6}}^{\frac{\pi}{6}} \,dt \\
  &= \frac{\pi}{3\sqrt{3}} \\
  \therefore \int_{0}^{1} \frac{dx}{1+x^3} &= \frac{1}{3}\log 2 +\frac{\pi}{3\sqrt{3}} \\
\end{flalign*}

\newpage

\begin{flalign*}
  I &= \int e^x \cos x \,dx \\
  &= e^x \sin x - \int e^x \sin x \,dx \\
  &= e^x \sin x - \left(e^x \cdot -\cos x - \int e^x \cdot -\cos x \,dx \right) \\
  &= e^x \sin x + e^x \cos x - \int e^x \cos x \,dx \\
  2I &= e^x \sin x + e^x \cos x \\
  I &= \frac{e^x}{2}\left(\sin x + \cos x\right) +C \\
\end{flalign*}

\end{document}