\documentclass[fleqn]{ltjsarticle}
\usepackage{amsmath}
\usepackage{amssymb}
\usepackage{tikz}
\usepackage{nccmath}
\usepackage{esint}
\usepackage{mleftright}
\everymath{\displaystyle}
\mleftright

\title{Integral(tmp)}
\author{Noot}
\date{04/20/2025}

\begin{document}

\begin{flalign*}
  \int \tan x \,dx \\
  \int_{-\sqrt{3}}^{\sqrt{3}} \sqrt{3-x^2}\,dx \\
  \int \log x \,dx \\
  \int_{\frac{\pi}{6}}^{\frac{\pi}{3}} \frac{\log \left(\sin x \right)}{\tan x} \,dx \\
  \int e^x \cos x \,dx \\
  \int \frac{dx}{\sin^3 x} \\
  \int \tan^4 x \,dx \\
  \int \frac{dx}{\sin^2 x} \\
  \int \sqrt{1-x} \,dx \\
  \int_{0}^{\frac{\pi}{2}} \cos^2 x \,dx \\
  \int x^2 \sin x \,dx \\
  \int_{0}^{2 \pi} \sqrt{1+ \cos x} \,dx \\
  \int_{0}^{\frac{\pi}{2}} \sqrt{\frac{1- \cos x}{1+ \cos x}} \,dx \\
  \int_{0}^{\frac{\pi}{3}} \sqrt{1+ \sin x} \,dx \\
  \int_{0}^{1} \frac{dx}{1+x^2} \\
\end{flalign*}
  
\newpage

\begin{flalign*}
  \int_{0}^{1} \frac{dx}{3+x^2} \\
  \int_{0}^{1} \frac{dx}{x^2+x+1} \\
  \int x^x \left(1+ \log x \right) \,dx \\
  \int \frac{dx}{\sqrt{1+x^2}} \\
  \int_{2}^{3} \frac{dx}{\sqrt{x^2-1}} \\
  \int \sqrt{1-e^{-2x}} \,dx \\
  \int_{0}^{1} \frac{dx}{1+x^3} \\
  \int \frac{dx}{\sin^4 x} \\
  \int \frac{dx}{\sqrt{x}+\sqrt{x+2}} \\
  \int_{0}^{e} \frac{e^x}{e^{e-x}+e^x} \,dx \\
  \int_{-1}^{1} \sqrt{1-x^2} \,dx \\
  \int \sqrt{1-e^{-2x}} \,dx \\
  \int \tan^2 x \,dx \\
  \int \tan^3 x \,dx \\
  \int \tan^5 x \,dx \\
\end{flalign*}

\newpage

\begin{flalign*}
  King Property \\
  \int_{a}^{b} f \left(x \right) \,dx &= \int_{a}^{b} f \left(a+b-x \right) \,dx \\
  t&=a+b-x \: と置換する \\
  \,dx &= - \,dt \\
  \therefore \int_{b}^{a} -f \left(t \right) \,dt &= \int_{a}^{b} f \left(t \right) \,dt \\
  &= \int_{a}^{b} f \left(x \right) \,dx \: \left(\because 定積分では変数を変更してもよい \right) \\
  因って、\int_{a}^{b} f \left(x \right) \,dx &= \int_{a}^{b} f \left(a+b-x \right) \,dx \\
\end{flalign*}

\begin{tabular}{|c|c|} \hline
  $x$ & $a \to b$ \\ \hline
  $t$ & $b \to a$ \\ \hline
\end{tabular}

\newpage

\begin{flalign*}
  \int \tan x \,dx &= \int \frac{\sin x}{\cos x} \,dx \: \left(\because \tan x = \frac{\sin x}{\cos x} \right) \\
  &= - \int \frac{- \sin x}{\cos x} \,dx \\
  &= - \log \left\lvert \cos x \right\rvert + C \: \left(\because \int \frac{f' \left(x \right)}{f \left(x \right)} \,dx = \log \left\lvert f \left(x \right) \right\rvert + C \right) \\
\end{flalign*}

\newpage

\begin{flalign*}
  \int_{- \sqrt{3}}^{\sqrt{3}} \sqrt{3-x^2} \,dx \\
  x &= \sqrt{3} \sin t \: とおく \\
  dx &= \sqrt{3} \cos t \,dt \\
  \int_{- \sqrt{3}}^{\sqrt{3}} \sqrt{3-x^2} \,dx &= \int_{- \frac{\pi}{2}}^{\frac{\pi}{2}} \sqrt{3- \left(\sqrt{3} \sin t \right)^2} \sqrt{3} \cos t \,dt \\
  &= 3 \int_{- \frac{\pi}{2}}^{\frac{\pi}{2}} \sqrt{1 - \sin^2 t} \cos t \,dt \\
  &= 3 \int_{- \frac{\pi}{2}}^{\frac{\pi}{2}} \left\lvert \cos t \right\rvert \cos t \,dt \\
  &= 3 \int_{- \frac{\pi}{2}}^{\frac{\pi}{2}} \cos^2 t \,dt \:\left(\because - \frac{\pi}{2} \leq t \leq \frac{\pi}{2} \Rightarrow 0 \leq \forall \cos t \right) \\
  &= \frac{3}{2} \int_{- \frac{\pi}{2}}^{\frac{\pi}{2}} 1+\cos 2t \,dt \:\left(\because \cos^2 x = \frac{1 + \cos 2x}{2} \:\:\:\: \, c.f. \sin^2 x = \frac{1 - \cos 2x}{2} \right) \\
  &= \frac{3}{2} \left[t + \frac{1}{2} \sin 2t \right]_{- \frac{\pi}{2}}^{\frac{\pi}{2}} \\
  &= 3 \left[t + \frac{1}{2} \sin 2t \right]_{0}^{\frac{\pi}{2}} \\
  &= \frac{3}{2} \pi \\
\end{flalign*}

\begin{tabular}{|c|c|} \hline
  $x$ & $- \sqrt{3} \to \sqrt{3}$ \\ \hline
  $t$ & $- \frac{\pi}{2} \to \frac{\pi}{2}$ \\ \hline
\end{tabular}

\newpage

\begin{flalign*}
  \int \frac{dx}{\sin^2 x} &= \int \frac{dx}{\cos^2 \left(\frac{\pi}{2} - x \right)} \:\left(\because \sin x = \cos\left(\frac{\pi}{2}-x \right) \right) \\
  &= - \tan\left(\frac{\pi}{2} - x \right) + C \\
  &= - \frac{1}{\tan x} + C \:\left(\because \tan x = \tan\left(\frac{\pi}{2}-x \right) \right) \\
\end{flalign*}

\newpage

\begin{flalign*}
  \int_{-1}^{1} \sqrt{1-x^2} \,dx \\
  x &= \sin t \: とおく \\
  dx &= \cos t \,dt \\
  \int_{-1}^{1} \sqrt{1-x^2} \,dx &= \int_{- \frac{\pi}{2}}^{\frac{\pi}{2}} \sqrt{1-\sin^2 t} \cos t \,dt \\
  &= \int_{- \frac{\pi}{2}}^{\frac{\pi}{2}} \left\lvert \cos t \right\rvert \cos t \,dt \\
  &= \int_{- \frac{\pi}{2}}^{\frac{\pi}{2}} \cos^2 t \,dt \:\left(\because - \frac{\pi}{2} \leq x \leq \frac{\pi}{2} \Rightarrow 0 \leq \forall \cos x \right) \\
  &= \int_{- \frac{\pi}{2}}^{\frac{\pi}{2}} \frac{1+\cos 2t}{2} \,dt \:\left(\because \cos^2 x = \frac{1 + \cos 2x}{2} \:\:\:\: \, c.f. \sin^2 x = \frac{1 - \cos 2x}{2} \right) \\
  &= \int_{- \frac{\pi}{2}}^{\frac{\pi}{2}} \left(\frac{1}{2}+\frac{\cos 2t}{2} \right) \,dt \\
  &= \frac{1}{2}\left[t+\frac{1}{2}\sin 2t \right]_{- \frac{\pi}{2}}^{\frac{\pi}{2}} \\
  &= \frac{\pi}{2}
\end{flalign*}

\begin{tabular}{|c|c|} \hline
  $x$ & $-1 \to 1$ \\ \hline
  $t$ & $- \frac{\pi}{2} \to \frac{\pi}{2}$ \\ \hline
\end{tabular}

\newpage

\begin{flalign*}
  \int_{0}^{1} \frac{\,dx}{1+x^2} \\
  x &= \tan t \: とおく \\
  \frac{1}{1+x^2} &= \frac{1}{\cos^2 t} \\
  dx &= \frac{dt}{\cos^2 t} \\
  \int_{0}^{1} \frac{\,dx}{1+x^2} &= \int_{0}^{\frac{\pi}{4}} \frac{\frac{1}{\cos^2 t}}{\frac{1}{\cos^2 t}} \,dt \\
  &= \int_{0}^{\frac{\pi}{4}} \,dt \\
  &= \frac{\pi}{4} \\
\end{flalign*}

\begin{tabular}{|c|c|} \hline
  $x$ & $0 \to 1$ \\ \hline
  $t$ & $0 \to \frac{\pi}{4}$ \\ \hline
\end{tabular}

\newpage

\begin{flalign*}
  \int \tan^4 x \,dx &= \int \tan^2 x \tan^2 x \,dx \\
  &= \int \tan^2 x \left(\frac{1}{\cos^2 x}-1 \right) \,dx \\
  &= \int \frac{\tan^2 x}{\cos^2 x} \,dx - \int \tan^2 x \,dx \\
  &= \frac{1}{3}\tan^3 x -\tan x + x +C \\
\end{flalign*}

\newpage

\begin{flalign*}
  \int \log x \,dx &= \int 1 \cdot \log x \,dx \\
  &= x\log x -\int x \cdot \frac{1}{x} \,dx \\
  &= x\log x -x +C \\
\end{flalign*}

\newpage

\begin{flalign*}
  \int_{0}^{e} \frac{e^x}{e^{e-x}+e^x} \,dx &= \int_{0}^{e} \frac{e^{e-x}}{e^x+e^{e-x}} \,dx \:\left(\because King Property \right) \\
  I &= \int_{0}^{e} \frac{e^x}{e^{e-x}+e^x} \,dx \: とすると \\
  2I &= \int_{0}^{e} \frac{e^x}{e^{e-x}+e^x} \,dx + \int_{0}^{e} \frac{e^{e-x}}{e^x+e^{e-x}} \,dx \\
  &= \int_{0}^{e} \frac{e^x+e^{e-x}}{e^x+e^{e-x}} \,dx \\
  &= \int_{0}^{e} \,dx \\
  &= e \\
  \therefore I &= \frac{e}{2} \\
\end{flalign*}

\newpage

\begin{flalign*}
  \int \frac{dx}{\sin^4 x} &= \int \frac{dx}{\tan^4 x \cos^4 x} \\
  &= \int \frac{1}{\tan^4 x} \left(1+\tan^2 x \right)\frac{dx}{\cos^2 x} \\
  t &= \tan x \: とおく \\
  dt &= \frac{dx}{\cos^2 x} \\
  \int \frac{1}{\tan^4 x} \left(1+\tan^2 x \right)\frac{dx}{\cos^2 x} &= \int \frac{1}{t^4} \left(1+t^2 \right) \,dt \\
  &= -\frac{1}{3t^3}-\frac{1}{t}+C \\
  &= -\frac{1}{3\tan^3 x}-\frac{1}{\tan x}+C \\
\end{flalign*}

\newpage

\begin{flalign*}
  \int \tan^3 x \,dx &= \int \tan x \cdot \tan^2 x \,dx \\
  &= \int \tan x \left(\frac{1}{\cos^2 x}-1 \right) \,dx \\
  &= \int \frac{\tan x}{\cos^2 x} \,dx - \int \tan x \,dx \\
  &= \frac{1}{2}\tan^2 x + \log \left\lvert \cos x \right\rvert +C \\
\end{flalign*}

\newpage

\begin{flalign*}
  \int \tan^2 x \,dx &= \int \left(\frac{1}{\cos^2 x}-1 \right) \,dx \\
  &= \tan x -x +C \\
\end{flalign*}

\newpage

\begin{flalign*}
  \int_{0}^{\frac{\pi}{2}} \cos^2 x \,dx &= \int_{0}^{\frac{\pi}{2}} \sin^2 x \,dx \\
  I &= \int_{0}^{\frac{\pi}{2}} \cos^2 x \,dx \\
  2I &= \int_{0}^{\frac{\pi}{2}} \cos^2 x + \sin^2 x \,dx \\
  &= \int_{0}^{\frac{\pi}{2}} \,dx \\
  &= \frac{\pi}{2} \\
  \therefore I &= \frac{\pi}{4} \\
\end{flalign*}

\newpage

\begin{flalign*}
  \int \frac{\log \left(\sin x \right)}{\tan x} \,dx &= \int \frac{\cos x \log \left(\sin x \right)}{\sin x} \,dx \\
  &= \int \frac{\log t}{t} \,dt \\
  &= \frac{1}{2}\left(\log t \right)^2 +C \\
  &= \frac{1}{2}\left(\log \left(\sin x \right) \right)^2 +C \\
\end{flalign*}

\newpage

\begin{flalign*}
  \int \frac{dx}{\sqrt{1+x^2}} \\
  x &= \sinh t \\
  dx &= \cosh t \,dt \\
  \int \frac{dx}{\sqrt{1+x^2}} &= \int \frac{1}{\sqrt{1+\sinh^2 t}}\cosh t \,dt \\
  &= \int \,dt \\
  &= t+C \\
  &= \log \left(x+\sqrt{1+x^2}\right)+C \\
\end{flalign*}

\newpage

\begin{flalign*}
  \int \tan^5 x \,dx &= \int \tan^3 x \left(\frac{1}{\cos^2 x}-1 \right) \,dx \\
  &= \frac{1}{4}\tan^4 x - \frac{1}{2}\tan^2 x + \log \left\lvert \cos x \right\rvert +C \\
\end{flalign*}

\newpage

\begin{flalign*}
  \int \frac{dx}{\cos^4} &= \int \frac{1}{\cos^2 x} \cdot \frac{1}{\cos^2 x} \,dx \\
  &= \int \left(1+\tan^2 x \right) \frac{dx}{\cos^2 x} \\
  &= \frac{1}{3}\tan^3 x + \tan x +C \\
\end{flalign*}

\newpage

\begin{flalign*}
  \int_{0}^{1} \frac{dx}{x^2+3} \\
  x &= \sqrt{3}\tan t \: とおく \\
  dx &= \frac{dt}{\cos^2 t} \\
  \int_{0}^{1} \frac{dx}{x^2+3} &= \int_{0}^{\frac{\pi}{6}} \frac{\frac{\sqrt{3}}{\cos^2 t}}{3\tan^2 t +3} \,dt \\
  &= \int_{0}^{\frac{\pi}{6}} \frac{\frac{\sqrt{3}}{\cos^2 t}}{3\left(\frac{1}{\cos^2 t}\right)} \,dt \\
  &= \frac{1}{\sqrt{3}} \int_{0}^{\frac{\pi}{6}} \,dt \\
  &= \frac{\pi}{6\sqrt{3}} \\
\end{flalign*}

\begin{tabular}{|c|c|} \hline
  $x$ & $0 \to 1$ \\ \hline
  $t$ & $0 \to \frac{\pi}{6}$ \\ \hline
\end{tabular}

\newpage

\begin{flalign*}
  \int x^2 \sin x \,dx \\
  \int x^2 \sin x \,dx &= x^2\sin x+2x\cos x -2\sin x+C \\
\end{flalign*}

\begin{tabular}{c c c}
  $+$ & $x^2$ & $\sin x$ \\
  $-$ & $2x$  & $-\cos x$ \\
  $+$ & $2$   & $-\sin x$ \\
\end{tabular}

\newpage

\begin{flalign*}
  \int \frac{dx}{\cos^3 x} &= \int \frac{\cos x}{\cos^4 x} \,dx \\
  &= \int \frac{\cos x}{\left(1-\sin^2 x \right)^2} \,dx \\
  &= \int \frac{dt}{\left(1-t^2 \right)^2} \: \left(t = \sin x \right) \\
  \frac{1}{\left(1-t^2 \right)} &= \frac{1}{2}\left(\frac{1}{1+t} + \frac{1}{1-t} \right) \\
  \left(\frac{1}{\left(1-t^2 \right)}\right)^2 &= \left(\frac{1}{2}\left(\frac{1}{1+t} + \frac{1}{1-t}\right) \right)^2 \\
  &= \frac{1}{4}\left(\frac{1}{\left( 1+t \right)^2} + \frac{1}{\left( 1-t \right)^2} + \frac{2}{1-t^2} \right) \\
  &= \frac{1}{4}\left(\frac{1}{\left(1+t\right)^2} + \frac{1}{\left(1-t\right)^2} + \frac{1}{1+t} + \frac{1}{1-t} \right) \\
  \int \frac{dt}{\left(1-t^2 \right)^2} &= \frac{1}{4} \int \left(\frac{1}{\left(1+t\right)^2} + \frac{1}{\left(1-t\right)^2} + \frac{1}{1+t} + \frac{1}{1-t} \right) \,dt \\
  &= \frac{1}{4} \left( \frac{1}{1+t} - \frac{1}{1-t} + \log \left\lvert \frac{1+t}{1-t} \right\rvert \right) +C \\
  &= \frac{1}{4} \left( \frac{1}{1+\sin x} - \frac{1}{1-\sin x} + \log \left\lvert \frac{1+\sin x}{1-\sin x} \right\rvert \right) +C \\
\end{flalign*}

\newpage

\begin{flalign*}
  \int \frac{dx}{\sin^4 x} \\
  t &= \tan x \\
  \cos^2 x &= \frac{1}{1+\tan^2 x} \\
  &= \frac{1}{1+t^2} \\
  \sin^2 x &= \cos^2 x \tan^2 x \\
  &= \frac{t^2}{1+t^2} \\
  dt &= \frac{dx}{\cos^2 x} \\
  dx &= \frac{dt}{1+t^2} \\
  \int \frac{dx}{\sin^4 x} &= \int \frac{\left( 1+t^2 \right)^2}{t^4} \cdot \frac{dt}{1+t^2} \\
  &= \int \frac{1+t^2}{t^4} \,dt \\
  &= \int \frac{dt}{t^4} + \int \frac{dt}{t^2} \\
  &= -\frac{1}{3t^3} - \frac{1}{t} +C \\
  &= -\frac{1}{3\tan^3 x} - \frac{1}{\tan x} +C
\end{flalign*}

\newpage

\begin{flalign*}
  \int_{0}^{2\pi} \sqrt{1+\cos x} \,dx &= \int_{0}^{2\pi} \sqrt{2\cos^2 \frac{x}{2}} \,dx \\
  &= \sqrt{2} \int_{0}^{2\pi} \left\lvert \cos \frac{x}{2} \right\rvert \,dx \\
  &= 2\sqrt{2} \left[\sin \frac{x}{2} \right]_{0}^{2\pi} \\
  &= 0 \\
\end{flalign*}

\newpage

\begin{flalign*}
  \int_{0}^{\frac{\pi}{3}} \sqrt{1+\sin x} \,dx &= \int_{0}^{\frac{\pi}{3}} \sqrt{1+\cos \left(\frac{\pi}{2}-x\right)} \,dx \\
  &= \sqrt{2} \int_{0}^{\frac{\pi}{3}} \left\lvert \cos \left( \frac{\pi}{4}-\frac{x}{2} \right) \right\rvert \,dx \\
  &= -2\sqrt{2} \left[ \sin \left( \frac{\pi}{4}-\frac{x}{2} \right) \right]_{0}^{\frac{\pi}{3}} \\
  &= -2\sqrt{2} \left( \frac{\sqrt{6}-\sqrt{2}}{4} - \frac{1}{\sqrt{2}} \right) \\
  &= 3-\sqrt{3} \\
\end{flalign*}

\end{document}