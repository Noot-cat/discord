\documentclass{ltjsarticle}
\usepackage{amsmath} 
\usepackage{tikz}

\begin{document}

a,bを$a^2+b^2<1$をみたす正の実数とする。また、座標平面上で原点を中心とする半径1の円をCとし、Cの内部にある2点 A(a, 0) , B(0, b) を考える。$ 0 < \theta  < \frac{\pi}{2} $に対してC上の点 P(cos \theta ,sin \theta) を考え、PにおけるCの接線に関してBと対称な点をDとおく。\\
(1) $ f(\theta ) = abcos 2 \theta  + asin \theta  - bcos \theta $とおく。方程式 f(\theta ) = 0 の解が$ 0 < \theta  < \frac{\pi}{2} $の範囲に少なくとも1つ存在することを示せ。\\
(2) Dの座標をb、\theta を用いて表せ。\\
(3) \theta が$ 0 < \theta < \frac{\pi}{2} $の範囲を動くとき、3点A, P, Dが同一直線上にあるような\theta は少なくとも1つ存在することを示せ。また、このような\theta はただ1つであることを示せ。\\
\\
\\
\\
\\
\\
pを3以上の素数とする。また、\theta を実数とする。\\
(1) cos3\theta とcos4\theta をcos\theta の式として表せ。\\
(2) $ cos\theta  = \frac{1}{p} $のとき,$ \theta = \frac{m}{n}*\pi $となるような正の整数 m, n が存在するか否かを理由を付けて判定せよ。

\end{document}