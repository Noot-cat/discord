\documentclass{ltjsarticle}
\usepackage{amsmath} 
\usepackage{tikz}

\begin{document}

二項定理とは何か、簡単に説明せよ。またシグマを使い$(a+b)^n$を展開せよ。\\
\\
nを使った式で表せ。\\
\begin{align}
  \sum_{k = 1}^{n} k  \\
  \sum_{k = 1}^{n} k^2 \\
  \sum_{k = 1}^{n} k^3 \\
  \sum_{k = 1}^{n} (2k-1) \\
  \sum_{k = 2}^{n+1} k  
\end{align}
\\
計算せよ。但し、積分定数をCとする。\\
\begin{align}
  \frac{d}{dx} (x^n) \text{(nは自然数)} \\
  \int x^n \,dx \text{(nは自然数)} \\
  \int_{1}^{3} 2x^3+6x^2-3 \,dx \\
  \int (5x^4+4^3+1) \,dx \\
  \frac{d}{dx} {\int (x^{1024}+5324x^{42})} \,dx \\
  \int (y^2+2y+1) \,dx 
\end{align}
\\
式(6)を証明せよ。\\
\\
(zの一次式)(zの高次多項式)の形で$z^5-1$を因数分解せよ。また、同様に$a^n-b^n$を因数分解せよ。\\
\\
数列$a_1=1,a_{n+1}=2a_n+1$の一般項を求めよ。\\
\\
集合$U=\{x | 1\leq x\leq500 \text{(xは整数)}\}$を全体集合とする。$\\A=\{a_n | a_1=1,a_{n+1}=2a_n+1,1\leq n \leq 5\},B=\{b_n | b_1=3,b_{n+1}=3b_n-2,1\leq n \leq 5\}$
について、$A \cap  B$を求めよ。 \\
\\
\\
tを実数とし、xy平面上の点P$(\cos 2t,\cos t)$および点Q$(\sin t,\sin 2t)$を考える。\\
(1)点Pと点Qが一致するようなtの値をすべて求めよ。\\
(2)tが$0 < t < 2\pi$の範囲で変化するとき,点Pの軌跡をxy平面上に図示せよ。ただし,x軸,y軸との共有点がある場合は,それらの座標を求め,図中に記せ。 

\end{document}