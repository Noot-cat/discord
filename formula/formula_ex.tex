\documentclass[fleqn]{ltjsarticle}
\usepackage{amsmath}
\usepackage{amssymb}
\usepackage{tikz}
\usepackage{nccmath}
\usepackage{esint}
\usepackage{mleftright}
\everymath{\displaystyle}
\mleftright

\begin{document}

\begin{flalign*} %関数_性質
  \cos^2 x + \sin^2 x &= 1 \\
  1-\sin^2 x &= (1+\sin x)(1-\sin x) \\
  &= \cos^2 x \\
  1-\cos^2 x &= (1+\cos x)(1-\cos x) \\
  &= \sin^2 x \\
  \tan x &= \frac{\sin x}{\cos x} \\
  1 + \tan^2 x &= \frac{1}{\cos^2 x} \\
  \tan^2 x &= \frac{1}{\cos^2 x}-1 \\
  \cos^2x&=\frac{1}{1+\tan^2x}\\
  \sin x &= 2\sin \frac{x}{2} \cos \frac{x}{2} \\
  \sin x \cos x&=\frac{1}{2}\sin 2x\\
  \cos^2x - \sin^2x &= \cos 2x \\
  &=(\cos x - \sin x)(\cos x + \sin x) \\
  &= 1- 2\cos^2 x \\
  &= 2\sin^2 x -1 \\
  \sin^2 x &= \frac{1-\cos 2x}{2} \\
  \cos^2 x &= \frac{1+\cos 2x}{2} \\
  \sin^{-1} x &= \arcsin x \\
  \cos^{-1} x &= \arccos x \\
  \tan^{-1} x &= \arctan x \\
  \sinh x &= \frac{e^x-e^{-x}}{2} \\
  \cosh x &= \frac{e^x+e^{-x}}{2} \\
  \tanh x &= \frac{e^x-e^{-x}}{e^x+e^{-x}} \\
  \cosh^2 x - \sinh^2 x &= 1 \\
\end{flalign*}

\newpage

\begin{flalign*} %逆関数
  x &= \sinh t \\
  e^t - e^{-t} -2x &= 0 \\
  e^{2t} -2e^tx -1 &= 0 \\
  e^t &= x+\sqrt{1+x^2} \: (\because e^t > 0)\\
  t &= \log \left(x+\sqrt{1+x^2}\right) \\
  x &= \cosh t \\
  e^t + e^{-t} -2x &= 0 \\
  e^{2t} +2e^tx -1 &= 0 \\
  e^t &= -x \pm \sqrt{1+x^2} \\
  t &= \log \left(-x \pm \sqrt{1+x^2}\right) \\
\end{flalign*}

\newpage

\begin{flalign*} %微分問
  y &= a \\
  y &= x^r \\
  y &= \sqrt{x} \\
  y &= \frac{1}{x} \\
  y &= \sin x \\
  y &= \cos x \\
  y &= \tan x \\
  y &= e^x \\
  y &= \log x \\
  y &= a^x \\
  y &= f \left(g \left(x \right) \right) \\
  y &= f \left(x \right)g \left(x \right) \\
  y &= \frac{f \left(x \right)}{g \left(x \right)} \\
  y &= x^x \\
  y &= \int_{0}^{x} f \left(t \right) \,dt \\
  x^2+y^2 &= 1 \\
  x &= f \left(y \right) \\
  \left\{ 
    \begin{alignedat}{2}   
      x &= f \left( t \right) \\
      y &= g \left( t \right) \\
    \end{alignedat} 
  \right.
\end{flalign*}

\newpage

\begin{flalign*} %積分問
  \int x^r \,dx \: \left(r \neq -1\right) \\
  \int \frac{1}{x} \,dx \\
  \int e^x \,dx \\
  \int \log x \,dx \\
  \int \sin x \,dx \\
  \int \cos x \,dx \\
  \int \tan x \,dx \\
  \int \frac{dx}{\cos^2 x} \\
  \int \frac{dx}{\sin^2 x} \\
  \int a^x \,dx \: \left(a>0, a \neq 1\right) \\
\end{flalign*}

\newpage

\begin{flalign*} % 微分解
  \frac{dy}{dx} &= 0 \\
  \frac{dy}{dx} &= rx^{r-1} \\
  \frac{dy}{dx} &= \frac{1}{2\sqrt{x}} \\
  \frac{dy}{dx} &= -\frac{1}{x^2} \\
  \frac{dy}{dx} &= \cos x \\
  \frac{dy}{dx} &= -\sin x \\
  \frac{dy}{dx} &= \frac{1}{\cos^2 x} \\
  \frac{dy}{dx} &= e^x \\
  \frac{dy}{dx} &= \frac{1}{x} \\
  \frac{dy}{dx} &= a^x\log a \\
  \frac{dy}{dx} &= f' \left(g \left(x \right) \right) \cdot g' \left(x\right) \\
  \frac{dy}{dx} &= f' \left(x \right)g \left(x \right) + f \left(x \right)g' \left(x \right) \\
  \frac{dy}{dx} &= \frac{f' \left(x \right)g \left(x \right) - f \left(x \right)g' \left(x \right)}{\left(g \left(x \right)\right)^2} \\
  \frac{dy}{dx} &= x^x\left(1+\log x\right) \\
  \frac{dy}{dx} &= f \left(x \right) \\
  \frac{dy}{dx} &= \frac{x}{y} \\
  \frac{dy}{dx} &= \frac{1}{\frac{d f\left(y\right)}{dy}} \\
  \frac{dy}{dx} &= \frac{\frac{dy}{dt}}{\frac{dx}{dt}} \\
\end{flalign*}

\newpage

\begin{flalign*} % 積分解
  \int x^r \,dx \: \left(r \neq -1\right) &= \frac{1}{r+1} x^{r+1} +C \\
  \int \frac{1}{x} \,dx &= \log x +C \\
  \int e^x \,dx &= e^x +C \\
  \int \log x \,dx &= x\log x -x +C \\
  \int \sin x \,dx &= -\cos x +C \\
  \int \cos x \,dx &= \sin x +C \\
  \int \tan x \,dx &= -\log \left\lvert \cos x \right\rvert +C \\
  \int \frac{dx}{\cos^2 x} &= \tan x +C \\
  \int \frac{dx}{\sin^2 x} &= -\frac{1}{\tan x} +C \\
  \int a^x \,dx \: \left(a>0, a \neq 1\right) &= \frac{a^x}{\log a} +C \\
\end{flalign*}


\end{document}